\chapter{Expected Outcomes}

We have proposed the use of sensing design approaches to support ill-posed problems of \emph{2-D} binary subsurface channels characterization. We will address this problem by incorporating side information of spatial correlations from training images and \emph{MPS} realizations.

Preliminary results give us some idea about the overall impact of proposed approaches. These approaches outperforms the \emph{MPS} outcomes based on \emph{SNESIM} algorithm and classical unstructured sampling. Classical \emph{MPS} tools provide only a wide set of plausible realizations of the wanted signal focus on honoring patterns in a training image. However, the proposed methods are focused on recovery actual structure from low acquisition regimes, providing a mechanism to improve the performance of \emph{MPS} simulations..

A key expected contribution rely on the formulation of the sensing problem and the adaption of an adaptive sensing strategy applied to the characterization of geological facies combined within the context of \emph{MPS}.

%On the one side, 
Our \emph{OWP} approaches use estimation of the \emph{pdfs} from \emph{MPS} realizations. The hypothesis regarding that \emph{MPS} methods generate realizations preserving high order statistics from training images conditioned to hard data allowing an empirical approximation of the statistical information about the field. We expect that our method improves classical \emph{MPS} simulation in the sense that the variability of realizations will be drastically reduced, as well as the bias with respect the true image.

In addition, in the current year we are consolidating the developed approaches writing a manuscript related with \emph{OWP} in \emph{2-D} binary channels of permeability. 

%On the other side, for \emph{NCS} approaches we propose the use of a set of thousands of realizations of the field by \emph{MPS} to calculate the statistical spatial correlation of the regionalized model. Therefore, we will to estimate covariance matrices. Thus, \emph{MPS} is proposed as an excellent information source to estimate spatial dependence of the actual field conditioned to the \emph{TI} and hard data. Additionally, several definitions of covariance approximations will be analyzed, in order to illustrate the relevance of a full or partial spatial interdependence characterization.









		
%We will study Adaptive Compressive Sensing (\emph{ACS}) theory and implement some approaches oriented to merge our \emph{OWP} principles and \emph{NCS} based reconstruction solvers in a general framework. As our previous implementations are focused on near optimal estimations we will work on some Near Optimal Adaptive Compressive Sensing (\emph{NoACS}) theoretical results and its possible use in our problem.

		
%		\item {Geological Binary Permeability Channels Images Reconstruction by Noisy Compressive Sensing: } We will to write a manuscript consolidating our work of \emph{NCS} for channelized structures.
%		\item {NoACS for Binary Permeability Channels: } We propose the beginning of a manuscript incorporating the ideas developed for \emph{NoACS} applied to regionalized variables.
