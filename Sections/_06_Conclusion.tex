\begin{conclusion}

This research plan proposes: (1) to formulate the optimal sampling strategy required for the characterization of \emph{2-D} regionalized fields considering the incorporation of side information related with spatial dependence, (2) to compare the proposed method with classical preferential and no-preferential sampling schemes, and (3) to reduce the variability of the \emph{MPS} realizations providing an useful tool for exploration and characterization tasks. In order to achieve these topics, we propose the use of both sampling design methods and the selection of \emph{ad hoc} inference process in a novel near optimal adaptive sensing approach. We will adapt, improve, and exploit prior information from training images for \emph{2D} channelized structures.

It is important to underline that there are not previous work that merge the topics covered in this proposal for  recovery or characterization of geological fields based on \emph{MPS} realizations. In addition, there are not previous comprehensive analysis developed in this kind of media that exploit spatial constrains in the way proposed. Therefore, from the expected outcomes for this research we could ultimately lead to novel methods of adaptive sensing and inference exploiting spatial constrains of binary regionalized fields. 

Although the proposed methods will be focused on \emph{2-D} binary channelized structures, the principles and application can be easily extended to other signals with spatial constraints. Depending on the achievement of proposed schedules, we would to explore the applicability of our results in others problems at the final stages of this thesis.



\end{conclusion}