\begin{abstract}

\emph{Geostatistic} tools have become the standard for characterizing the spatial distribution of geological subsurface structures. However, the problem of image recovery for regimes with low acquisition rates still poses a complex issue. In the last decade, several alternative methods for experimental design at low sampling rates were developed providing insights into the use of additional prior information to achieve better performance in the reconstruction and characterization of geological images. Based on these achievements, a new challenge is to incorporate tools from the \emph{state of art} in signal processing and stochastic modeling to improve this kind of inference problems. We propose a comprehensive study of inverse problems at low sampling rates with strong focus on Geosciences and, in particular, for the reconstruction of binary permeability channels.

%Based on these achievements, the challenge is to incorporate several tools from the \emph{state of art} in signal processing/reconstruction and stochastic modeling to improve this kind of inference problems with special interest in geoscientific applications. We propose a comprehensive study of inference problems at low-rate sampling with strong focus in Geosciences (binary permeability channels).

In this work, we will work on the formulation and experimental analysis of the \emph{Optimal Well Placement} (\emph{OWP}) problem. This problem tries to find the best way of distributing the measurements (or samples) to optimize sensing/locating resources in areas of mining and drilling. This work aims at formalizing the problem of the \emph{OWP} for a given amount of available measurements. The characterization of the uncertainty will be a central piece of this formalization. In particular, the \emph{OWP} problem will be addressed from the perspective of minimizing the remaining field uncertainty and sequential algorithms will be proposed to solve it. We conjecture that \emph{OWP} based locations will be distributed on transition zones of binary fields, and then we will assess this behavior with respect to standard sensing schemes.

We will study the use of \emph{information theoretic} concepts such as \emph{conditional entropy} to characterize the uncertainty related to a geological model conditioned to the acquisition of data (well logs), and its application in a preferential sampling strategy oriented to improve geostatistical inference at low acquisition rates. 

In the experimental side, a greedy sequential algorithm will be proposed to approximate the \emph{Information Theoretic} (\emph{IT}) based \emph{OWP} sampling to show this principle. \emph{IT-OWP} provides MPS realizations with reduced variability for geological binary facies models in the critical low sampling regime.

Finally, we will study the performance of different inference processes under the proposed sampling schemes, as well as to evaluate our proposed method with different techniques presented in the literature. 

%For the systematic analysis concerning the use and integration of \emph{MPS} and training images, a -UNIQUE- model of the
%--MOST PERTINENT ACTUAL CASE-- will be investigated, which would shows a --RELEVANT EXPECTED OUTCOME--- those --RESEARCH TOPIC- that influence the --RELEVANT INFLUENCE AREA--. As some of these --RESEARCH TOPIC- are --REVELANT COMMENTARY

% Sgems is used to create appropiate ,,,
% TIPS is developed as a ....

%Furthermore, --RELEVANT TOPOC- in the process of --RELEVANT PROCCES-- will be investigated, where the focus
%will be on a comparison of ----- as opposed to ----



\end{abstract}